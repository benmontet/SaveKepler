\documentclass[12pt]{article}
\usepackage{enumitem, color, hyperref}
\definecolor{hypercolor}{RGB}{0,0,127}
\hypersetup{%
  citecolor=hypercolor,%
  linkcolor=hypercolor,%
  urlcolor=hypercolor%
}%

\newcommand{\sectionname}{Section}
\newcommand{\documentname}{\textsl{White Paper}}
\newcommand{\foreign}[1]{\textit{#1}}
\newcommand{\vs}{\foreign{vs}}
\newcommand{\observatory}[1]{\textsl{#1}}
\newcommand{\kepler}{\observatory{Kepler}}
\newcommand{\Kepler}{\kepler}

\begin{document}

\section*{Maximizing \Kepler\ two-wheel science return per telemetered pixel by building detailed image models}
\noindent
A white paper submitted in response to the \Kepler\ Project Office
\textit{Call for White Papers: Soliciting Community Input for
  Alternate Science Investigations for the Kepler
  Spacecraft}\footnote{\url{http://keplergo.arc.nasa.gov/docs/Kepler-2wheels-call-1.pdf}}
released 2013 August 02.

\begin{description}[style=nextline,itemsep=0ex]
\item[David W. Hogg]
\textit{Center for Cosmology and Particle Physics, New York University}
\item[Tom Barclay]
\textit{NASA Ames Research Center}
\item[Rebekah Dawson]
\textit{Havard--Smithsonian Center for Astrophysics}
\item[Rob Fergus]
\textit{Courant Institute of Mathematical Sciences, New York University}
\item[Dan Foreman-Mackey]
\textit{Center for Cosmology and Particle Physics, New York University}
\item[Michael Hirsch]
\textit{Max-Planck-Institut f\"ur Intelligente Systeme}
\item[Dustin Lang]
\textit{McWilliams Center for Cosmology, Carnegie Mellon University}
\item[Ben Montet]
\textit{Department of Astronomy, California Institute of Technology}
\item[David Schiminovich]
\textit{Department of Astrophysics, Columbia University}
\item[Bernhard Sch\"olkopf]
\textit{Max-Planck-Institut f\"ur Intelligente Systeme}
\end{description}

\clearpage

\section{Executive summary}

\paragraph{primary recommendation:}
Fundamentally, this \documentname\ advocates \emph{image modeling}---%
  building a detailed model of the pixel sensitivities and point-spread function
  as a function of focal-plane position,
  along with a model of the position and brightness of every star in the field.
This level of modeling has not happened with \Kepler\ up to now
  because the data have been made extremely precise with good pointing
  and aperture photometry.
In the two-wheel era, \Kepler\ will not maintain precise pointing.
This is a benefit as well as a curse:
It reduces the precision of naive aperture photometry,
  but it provides data diversity that permits inference
  of the sensitivity map and point-spread function
  that is not perfectly covariant with the morphology of the true scene.
We propose capitalizing on this to
  \textbf{develop a probabilistic generative model of the \Kepler\ pixels.}
We argue that this modeling will permit continuance of photometry at 10-ppm-level precision.
In \sectionname~\ref{sec:extant} we show that we can improve \Kepler\ photometric precision with a focal-plane model.
In \sectionname~\ref{sec:future} we show that the expected drift or jitter in positions in the two-weel era
  will \textsl{(a)}~help with constraining this kind of model,
  and \textsl{(b)}~be obviated (in terms of loss of precision) by such modeling.
These results are relevant to \emph{almost any} scientific goal for the repurposed mission,
  independent of our secondary recommendations.

\paragraph{secondary recommendation 1:}
Shorten exposure times and increase cadence.
Note value for TTVs and asteroseismology.

\paragraph{secondary recommendation 2:}
If we are right that current mission precision can be maintained in the two-wheel era,
  then it is our (admittedly debatable) view that the highest-impact scientific project for the \Kepler\ spacecraft
  remains in the area of exoplanet detection and characterization.
In particular, we advocate de-scoping the mission to focus on
  \textbf{finding Earth-like planets on year-ish orbits around Sun-like stars}.
This de-scope would involve
  \textsl{(a)}~reducing the \Kepler-field target list to increase the fraction of Sun-like stars
  (and reduce the total number of telemetered sources, compensating for the cost of higher cadence),
  \textsl{(b)}~removing from the target list stars that are known to be stochastically variable
  at a level that would render unlikely the discovery of an Earth-like planet
  (to further reduce number of telemetered sources),
  and \textsl{(c)}~continuing observations in the current \Kepler\ field
  (to extend the observations beyond four years).
That is, we specifically make the (radical) proposal to continue observations in the current \Kepler\ field,
  despite the stellar positional drift we expect for such observations.
As of writing, there is talk of increased pointing precision for fields with particular auspicious Sun angles.
If using such fields becomes a requirement of the repurposed mission,
  this would recommend against making year-ish orbit planets a high priority.

\section{Philosophy and motivation}

Image modeling will deliver many bits; we will get more science per
downloaded pixel than the naive predictions of degradation.

Kepler core goals remain outrageously interesting.

We know a lot about the prevalance of planets and the stars in which
we could possibly detect them; we should use this to reduce our data
volume.

TESS is coming; we can help with that.

Asteroseismology and transit timing variations both benefit from
shorter cadence.

\section{Modeling of extant \Kepler\ data}\label{sec:extant}

\section{Future data \vs\ extant data}\label{sec:future}

\section{Target selection}\label{sec:target}

\section{Spacecraft pointing}\label{sec:pointing}

\section{Aperture selection for telemetry}\label{sec:telemetry}

\end{document}
